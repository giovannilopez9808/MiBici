\section{Métodos y materiales}

\subsection{Base de datos}

\subsection{Promedio y varianza mensual por hora}

Sea un conjunto $Q(m,h)$ de datos para un mes $m$  entre las horas $h$ y $h+1$ con elementos $q$, entonces el promedio mensual por hora se calcula como:

\begin{equation}
    \mu_{Q(m,h)} = \frac{1}{n(m,h)} \sum_{i=1}^{n(m,h)} q_i \label{eq:monthly_hourly_mean}
\end{equation}

y la varianza se calcula como:

\begin{equation}
    \sigma^2_{Q(m,h)} = \frac{1}{n(m,h)-1} \sum_{i=1}^{n(m,h)} (q_i - \mu_{Q(m,h)})^2 \qquad \begin{matrix}
        m=1,2,\dots,12 \\
        h=6,7,\dots,23
    \end{matrix} \label{eq:monthly_hourly_var}
\end{equation}

donde $n(m,h)$ es el número total de elementos del conjunto $Q(m,h)$.

\subsection{Promedio y varianza diaria semanal por hora}

Sea un conjunto $Q(d,h)$ de datos para un día de la semana $d$  entre las horas $h$ y $h+1$ con elementos $q$, entonces el promedio diario semanal por hora se calcula como:

\begin{equation}
    \mu_{Q(d,h)} = \frac{1}{n(d,h)} \sum_{i=1}^{n(d,h)} q_i \label{eq:daily_hourly_mean}
\end{equation}

y la varianza se calcula como:

\begin{equation}
    \sigma^2_{Q(d,h)} = \frac{1}{n(d,h)-1} \sum_{i=1}^{n(d,h)} (q_i - \mu_{Q(d,h)})^2 \qquad \begin{matrix}
        d=1,2,\dots,7 \\
        h=6,7,\dots,23
    \end{matrix} \label{eq:daily_hourly_var}
\end{equation}

donde $n(d,h)$ es el número total de elementos del conjunto $Q(d,h)$.